\chapter{Sažetak}

Kroz ovaj projekt smo uspoređivali osnovni konvolucijski model s modelom ResNet-18. Glavna razlika između ovih mrežnih arhitektura je postojanje rezidualnih preskočnih veza između slojeva. Njihova uloga je da u sljedeći sloj prenose ulaz trenutnog sloja, koji nije bio transformiran prolaskom kroz sloj. 

Istražili smo kako na uspješnost oba modela utječu različiti faktori poput veličine grupe, odabira različitih optimizatora te upotreba različitih funkcija gubitka. 

Modeli su trenirani kroz 30 epoha na skupovima podataka CIFAR i MNIST, te je njihova uspješnost evaluirana na nezavisnom skupu.