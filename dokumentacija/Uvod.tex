\chapter{Uvod}

Računalni vid je područje umjetne inteligencije koje se bavi interpretacijom slika. Iako su zadaci računalnog vida, poput raspoznavanja objekata na slici, vrlo često trivijalni za čovjeka, gotovo nemoguće je osmisliti algoritme kojima bi računala mogla obavljati isti posao. Stoga se računalni vid oslanja na metode dubokog učenja, koje se zadnjih par godina sve brže razvijaju zbog dostupnosti računalne snage. Kako tehnologija napreduje, možemo očekivati sve veću upotrebu računalnog vida u raznim domenama - od samovozećih automobila do raspoznavanja rukopisa. 

U gotovo svim primjenama nam je bitno da su naši modeli precizni, stoga u ovom radu razmatramo učinak nekoliko faktora na uspješnost modela - veličina grupe, optimizatori i funkcije gubitka. Ispravan odabir nekih od ovih faktora nam može učiniti naš model preciznijim.

Učinak spomenutih faktora na naše modele smo eksperimentalno istražili na dva skupa podataka - MNIST i CIFAR, nad kojima je zadatak našeg modela klasificirati objekt na slici. 