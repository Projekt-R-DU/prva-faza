\chapter{Vanjske biblioteke}

\section{Python}

U ovom smo projektu koristili Python, interpreterski programski jezik opće namjene i visoke razine. Jedan je od najprimjenjenijih jezika u strojnom i dubokom učenju te postoji velik broj biblioteka za tu namjenu. Za ostvarenje projekta koristili smo biblioteke: PyTorch, NumPy, Matplotlib 

\section{PyTorch}

PyTorch je biblioteka otvorenog koda za strojno učenje temeljena na Torch biblioteci, a razvio ju je Facebook (Facebook's AI Research lab). Primjenjuje se u računalnom vidu i obradi prirodnog jezika. PyTorch nudi mogućnost rada s tenzorima uz upotrebu grafičkog procesora u cilju poboljšanja performansi te rad s dubokim neuronskim mrežama rađenim na metodi automatske diferencijacije. 

\section{NumPy}

NumPy je biblioteka otvorenog koda za programski jezik Python koja omogućuje rad s višedimenzionalnim nizovima te velik broj matematičkih funkcija za upravljanje tim nizovima. Razlog velike popularnosti NumPy biblioteke je njena efikasnost i jednostavnost korištenja, a zbog rada s vektorima i matricama primjenjuje se u rješavanju problema strojnog učenja.

\section{Matplotlib}

Matplotlib je biblioteka otvorenog koda za prikaz statičkih, animiranih i interaktivnih vizualizacija u Pythonu nadahnuta MATLABom. Nudi širok raspon mogućnosti za crtanje 2D grafova i slika te se lako koristi s bibliotekom NumPy zbog čega se često koristi u strojnom učenju primjerice za vizualizaciju preciznosti modela.